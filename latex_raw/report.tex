\PassOptionsToPackage{quiet}{fontspec} 
\documentclass{ctexart}
\usepackage{lipsum}
% \ctexset{section={format+=\raggedright}}
\begin{document}
\title{数字图像处理基础-图书ISBN号字符识别}
\author{覃梓鑫(软工2003-20202005175)}
\date{\today}
\maketitle
\tableofcontents
\newpage
\section{概述}
\noindent
\textbf{设计目的:}\\
\textbf{内容:}\\
\textbf{运行环境:}
Windows10 + Python 3.10.6\\
所需 Python 第三方库如下:
\begin{itemize}
    \item 略
\end{itemize}
\noindent
\textbf{开发工具:}%不用加多余的\\
\begin{itemize}
    \item 操作系统 Windows 10 21H2
    \item 集成开发环境 Visual Studio Code 1.73.1
    \item 文档编写工具 TeXworks 0.6.6
    \item 编程语言 Python 3.10.6
    % \item 版本管理工具 git 2.29.0
    \item 编码格式 UTF8
\end{itemize}

\section{整体设计}
\section{具体实现}
\subsection{流程图}
\subsection{数学模型}
\subsubsection{灰度化}
根据课件(第11章-P38页-4彩色平衡)内容可知,彩色图像数字化后,景物颜色会偏移真实颜色,导致三基色不平衡。这里采用白平衡法计算灰度,即使用公式:
\[I(x,y)=0.299\cdot f_R(x,y)+0.587\cdot f_G(x,y)+0.114\cdot f_B(x,y)\]
\subsection{核心程序}
\subsection{处理过程}
\section{实验结果及分析}
\section{总结与体会}
\section{致谢}
\section{参考文献}
\section{附录}
\end{document}